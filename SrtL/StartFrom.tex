\lstinputlisting[style=Ebnf, firstline=18, lastline=19]{SrtL.ebnf}

A test is not required to start from the entry rule of a syntax. This is very 
useful when testing specific parts of a syntax. To define a starting point for a
specific test use the \textbf{start from} keywords followed by a 
\textit{string}. The string is the name of the rule to start testing a syntax 
from.

Because of this, no boiler plate input needs to be specified and therefore
there is no need to keep heaps of boiler plate input in sync with the rest of 
the syntax. In effect, the only tests that may need to be modified after the 
syntax changes are the tests that concerns the changed part of a syntax.

Notice that the referenced rule name is case sensitive. If the referenced rule 
does not exist in a syntax, an error must be issued.

\begin{lstlisting}[style = SrtL]
test
    description
        "Start the test somewhere else than the default starting rule."
    input 
        "Some input"
    start from
        "rule"
    is valid
\end{lstlisting}