\lstinputlisting[style=Ebnf, firstline=33, lastline=57]{SrtL.ebnf}

Strings are used to represent data such as rule names and test input. Strings 
consists of zero or more characters enclosed in double quotes, as in "hello", 
and may include both simple escape sequences and Unicode escape sequences.
The four hex digits in \textit{unicode code point} denotes a Unicode code point.
If \textbackslash prefixes anything else then the characters in table 
\ref{table:escape-sequence-translation}, an error must be issued.

\begin{table}[h]
    \centering
    \caption{Escape Sequence Translation}
    \label{table:escape-sequence-translation}
    \begin{tabular}{|l|l|}
        \hline
        \textbf{Escape Sequence}     & \textbf{Unicode Code Point}  \\ \hline
        \textbackslash"              & \textbackslash u0022         \\ \hline
        \textbackslash\textbackslash & \textbackslash u005C         \\ \hline
        \textbackslash 0             & \textbackslash u0000         \\ \hline
        \textbackslash a             & \textbackslash u0007         \\ \hline
        \textbackslash b             & \textbackslash u0008         \\ \hline
        \textbackslash f             & \textbackslash u0014         \\ \hline
        \textbackslash n             & \textbackslash u000A         \\ \hline
        \textbackslash r             & \textbackslash u000D         \\ \hline
        \textbackslash t             & \textbackslash u0009         \\ \hline
        \textbackslash v             & \textbackslash u000B         \\ \hline
    \end{tabular}
\end{table}

Multiline input is often required, however strings can not span multiple lines. 
The reason is that it is very hard to tell Windows style new lines and Linux 
style new lines apart without a hex editor. So SrtL forces the test writer to be
more explicit by using escape sequences or Unicode code points combined with 
concatenation. Notice that the "+" operator is not used to indicate string 
concatenation as it is in most programming languages. In SrtL, there is no need
for such operator. Here is an example of string concatenation.

\begin{lstlisting}[style = SrtL]
"Abc "
"def"
\end{lstlisting}

Is equal to 

\begin{lstlisting}[style = SrtL]
"Abc def"
\end{lstlisting}

The \textit{string list} represents a list of strings. The comma character 
separates each string. \textit{string list} has no support for concatenation.